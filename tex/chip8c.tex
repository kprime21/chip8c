\documentclass{article}
\title{\vspace{-7em} Chip8 Emulator in C}
\author{Kirjan Kohuladas/kprime21}
\date{}

\begin{document}
\maketitle


\section*{Overview}
Compact Hexadecimal Interpretive Programming – 8-bit 
\subsection*{Components}
\begin{itemize}
	\item Memory - 4kB ram (4096 bytes - 4096 address lines each line is 1 byte)
	\item Display - 64 x 32 pixels 
	\item Registers
		\begin{itemize}
			\item Program Counter (16 bits)
			\item Index Register (16 bits) 
			\item Stack - call subroutines and functions (16 bits)
			\item Delay timer - decremented at 60Hz (8 bits)
			\item Sound timer - decremented at 60Hz (8 bits)
			\item 15 General purpose registers - V0 - VF (8 bits)
		\end{itemize}
\end{itemize}
\subsection*{Memory}
\begin{itemize}
	\item all memory is RAM, 4096 bytes. 
		\begin{itemize}
			\item 4096 addressable lines
			\item 12 bits needed
			\item each addressable line represents an addresss of 1 byte.
		\end{itemize}
	\item interpreter located 0x000-0x1FF (not in our case)
	\item program located 0x200 - 0x...
	\item font located before program 0x000-0x1FF (popular area - 0x050 - 0x09F)
\end{itemize}
\subsection*{Font}
\begin{itemize}
	\item font character should be 4px x 5px
	\item first byte is the character (draw vertically in nvim to see)
	\item stored in memory, index register set to specific font in memory to draw it
\end{itemize}
\subsection*{Display}
\begin{itemize}
	\item 60Hz - 60 times per second 
\end{itemize}
\subsection*{Main}
\begin{itemize}
	\item Read through emulator guide
\end{itemize}
\subsection*{To Do}
\begin{itemize}
	\item Setup directory to ignore makefile outputs
\end{itemize}

\subsection*{Finished}
\begin{itemize}
	\item Set up SDL libray in C file
	\item Set up make file for compilation
\end{itemize}


\section*{Reference}
\begin{itemize}
	\item https://tobiasvl.github.io/blog/write-a-chip-8-emulator/
\end{itemize}

\end{document}
